\section{\label{sec:nota_controle}Notas sobre os controle de instrumentos}
O  \ml � um pacote comercial vers�til que agrega desde rotinas para manipula��o e processamento num�rico � rotinas para comunica��o entre computadores e \textit{hardware} de terceiros. O \ml � constru�do de forma modular, na qual recursos s�o incorporados na forma de \textit{toolboxes} relativamente independentes e 
 Neste curso iremos explorar esta ferramenta para automatizar a aquisi��o de dados atrav�s do controle dos instrumentos de medi��o e processamento dos dados obtidos.   Em particular utilizaremos extensivamente o \href{http://www.mathworks.com/help/instrument/index.html}{\ic}.

\section{Comandos b�sicos}
\subsection{Opera��es matem�ticas}
O console de comandos do \ml funciona como uma calculadora, pode-se definir vari�veis escalares ou matriciais e realizar opera��es matem�ticas com as mesmas. A sequ�ncia abaixo ilustra algumas opera��es simples, note que o s�mbolo (>>) aparece automaticamente no console e n�o dever ser digitado.  Para uma descri��o mais detalhada, vejo na p�gina sobre , \href{http://www.mathworks.com/help/matlab/matlab_env/enter-statements-in-command-window.html}{entrada de vari�veis}
\subsection{Escalares}
\begin{lstlisting}
>> a=2
>> b=3
>> a*b
ans =
     6
>> a+b
ans =
     5
>> a/b
ans =
    0.6667
\end{lstlisting}
\subsection{Matrizes}
Veja a p�gina de ajuda sobre \href{http://www.mathworks.com/help/matlab/matlab_prog/create-numeric-arrays.html}{constru��o de vetores e matrizes}. 
\subsection{Cadeia de caract�res - \textit{strings}}
Veja a p�gina de ajuda sobre \href{http://www.mathworks.com/help/matlab/matlab_prog/creating-character-arrays.html}{constru��o de \str}. 
\begin{lstlisting}
%% Connect to instrument and print its ID
delete (instrfindall); % Apaga qualquer instrumento que fora criado anteriormente
vinfo = instrhwinfo('visa', 'NI', 'USB');  %Descobre todos os instrumentos conectados ao computador atrav�s da interface USB        
\end{lstlisting}
\section{Comandos}
a
 
